
% introduction

%\begin{frame}[Overlay-Aktionen][Optionen]{Titel}{Untertitel}
%	Inhalt
%\end{frame}

\section{Introduction}

\begin{frame}{Programmieren}{Lizenz}

	\begin{block}{Creative Commons Lizenzvertrag}
     Dieses Werk bzw. Inhalt steht unter einer Creative Commons Namensnennung - Nicht-	kommerziell - Weitergabe unter gleichen Bedingungen 3.0 Unported Lizenz.
	\end{block}
	\begin{center}
	\href{http://creativecommons.org/licenses/by-nc-sa/3.0/deed.de}{$\blacktriangleright$ CC BY-NC-SA}
	\end{center}
\end{frame}


\begin{frame}{Programmieren}{Eine Einführung}
	Was lernt man hier:
	\begin{itemize}
	\item Wie ein Computer arbeitet
	\item Wie Daten in einem Computer aufgebaut sind
	\item Wie Programmiersprachen funktionieren
	\end{itemize}
	\vspace*{0.3cm}
	
	Worauf wird hier nicht näher eingegangen
	\begin{itemize}
	\item Eine echte Programmiersprache \todo{entfernen?}
	\item Genaue technische Details
	\item Warum? -> Später mehr
	\end{itemize}
\end{frame}


\begin{frame}{Programmieren}{Voraussetzungen}
	Zielgruppe: 12+ / ab 7. Klasse

	Notwendig:
	\begin{itemize}
	\item Interesse (auch ein ganz kleines bisschen für Mathe)
	\end{itemize}
	\vspace*{0.3cm}
	
	Von Vorteil:
	\begin{itemize}
	\item Englisch-Grundkenntnisse
	\item Verständnis für Mathematik
	\end{itemize}
\end{frame}

\begin{frame}{Programmieren}{Voraussetzungen}
	Mathematische Thematiken:
	\begin{itemize}
	\item Grundrechenarten ( Addieren, Subtrahieren, Multiplikation)
	\item Zahlensysteme (Dezimal, Binär/Dual-System)
	\item Potenzen ( $2^3$ )
	\end{itemize}
	
	
	Englisch:
	\begin{itemize}
	\item Nur einzelne Wörter
	\item http://dict.leo.org/
	\item http://www.dict.cc/
	\end{itemize}	
\end{frame}


\begin{frame}{Programmieren}{Motivation}
	Muss man das so machen?
		-> Nein
	
	Warum dann aber so detailliert?
		Es geht um das verstehen -> keine Magic Box
		Entmystifizierung des PCs
	
	Problematik:
		Kann am Anfang langweilig sein
		Kann am Anfang kompliziert klingen
			(Ist teilw. kompliziert)
	
	Persönliche Meinung:
		Bringt unheimlich viel
		Muss es nicht alles auswendig wissen
		Grobes Verständnis was im Hintergrund passiert
		Gut gerüstet für alle Programmiersprachen
	
\end{frame}


\begin{frame}{Programmieren}{Background}
	Was ist mein Ziel/Ansprüche
		-> Verständnis Arbeitsweise des PCs
	
	Kritik/Feedback/Anregungen
	
\end{frame}