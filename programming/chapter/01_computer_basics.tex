% computer basics
\section{Basics}

% how a computer works

\begin{frame}{Programmieren}{Wie funktioniert ein Computer}

	Um programmieren zu können, ist es ein Riesen vorteil wenn 
	man versteht wie ein Computer überhaupt funktioniert

 \begin{columns}[t]
     \begin{column}[T]{5cm} 
	    Vorab:
		\begin{itemize}
		\item Zahlensysteme
		\item Kern-Komponenten eines Computers
		\end{itemize}
     \end{column}
     \begin{column}[T]{5cm} 
        Wie arbeitet ein Computer:
		\begin{itemize}
		\item Befehle
		\item Speicher
		\item Variablen \& Addressen
		\end{itemize}
     \end{column}
     \end{columns}
     
\end{frame}


% Zahlen Systeme

\begin{frame}{Programmieren}{Zahlensysteme}

	Was ist ein Zahlensystem

	Wir rechnen im Dezimal System:
	\begin{itemize}
	\item 0,1,2,3,4,5,6,7,8,9,10 
	\item Ab dem Schritt 10  eine Stelle ergänzt(von 9 auf 10) 
	\item Wiederholt sich
	\item 10 Zahlen (0-9) also ein Zehnersystem
	\end{itemize}
	
	Das selbe funktioniert auch mit anderen Zahlensystemen zum Beispiel mit einem 3er-System
	\begin{itemize}
	\item Es gibt nur die Zahlen 0, 1, 2
	\item 3 Zahlen also ein 3er System
	\item 0, 1, 2, 10
	\item 11, 12, 20
	\item 21, 22, 30
	\end{itemize}
	
	\todo{Übungsaufgaben Zahlensystem? }
	
	Warum erzähl ich das?
	\begin{itemize}
	\item Computer funktionieren mit einem anderen Zahlensystem
	\item Binäres Zahlensystem
	\end{itemize}
	
\end{frame}

%binäre zahle 1
\begin{frame}{Programmieren}{Binäre Zahlendarstellung}

	Der Computer benutzt eine darstellung mit genau 2 Zahlen 0 oder 1 
	Da das ganze elektrisch funktioniert: Strom an oder Strom aus	
	
	0 oder 1
	\emph{an} oder \textit{aus}

	Dualsystem, Zweiersystem	
	
	das ganze fällt unter Digitaltechnik
	Wenn Strom eine andere rolle spielt: Analog
	
\end{frame}

%analog: https://de.wikipedia.org/wiki/Analogrechner

%binäre zahle 2
\begin{frame}{Programmieren}{Binäre Zahlendarstellung}

		Überlegen wir jetzt nochmal wie ein Zahlen system funktioniert:
		
		0, 1, 10, 11, 101
		
		Eine einzelne 0 oder 1 nennt man \textbf{Bit}


		Wieviele \textit{Zahlen} man darstellen kann:

		\begin{tabular}{|l||c|c|}
		\hline   & Binär & Dezimal \\ 
		\hline 1 & 0 & 0 \\ 
		\hline 2 & 1 & 1 \\ 
		\hline 3 & 10 & 2 \\ 
		\hline 4 & 11 & 3 \\ 
		\hline 5 & 100 & 4 \\ 
		\hline 6 & 101 & 5 \\ 
		\hline 7 & 110 & 6 \\ 
		\hline 8 & 110 & 7 \\
		\hline 
		\end{tabular} 
		
		Schauen wir das an;
		1 Bit $\rightarrow$ 2 Zahlen (0,1)
		2 Bit $\rightarrow$ 4 Zahlen (0-3)
		3 Bit $\rightarrow$ 8 Zahlen ( 0-7)
		

\end{frame}

\begin{frame}{Programmieren}{Binäre Zahlendarstellung}

		Das sind also immer
		
		$2^N$ N = Anzahl der Bits also $ 2^3 \rightarrow 2 * 2 * 2 = 8 $
		Zweierpotenz
		
		Dran denken 0 ist auch eine Zahl
		
		$2^8 ( 8 Bit -> 1 Byte) 256 Zahlen ( 0-255)$
		
		Namen:
		
		1 Bit
		1 Byte -> (8 Bit)
		1 KB -> ( 1024 Byte)
		1 MB -> ( 1024 KB) -> (1048576 Byte)
		1 GB
		....
		
		1000 vs 1024 
		1024 ist eine zweier Potenz
		
		$2^10$
		
		Häufig 2 Potenzen ->
		
		32 Bit -> 4 Byte -> $2^2$
		64 Bit -> 8 Byte -> $2^3$
		
%		http://de.wikipedia.org/wiki/Byte#Vergleich
		
		\todo{Übungsaufgaben Größe von 32 Bit etc? }
\end{frame}


\begin{frame}{Programmieren}{Binäre Zahlendarstellung}

		Zum umrechnen von Dezimal Zahlen zu Binär-Zahlen und umgekehrt
	
		\begin{tabular}{cccccccc|c}
		128 & 64 & 32 & 16 & 8 & 4 & 2 & 1 & Dezimal \\ \hline 
		0   & 0  & 0  & 0  & 0 & 0 & 0 & 1 & 1 \\ 
		0   & 0  & 0  & 1  & 0 & 1 & 0 & 1 & $16 + 4 + 1 = 21$ \\ 
		\hline 
		\end{tabular} 	
		
		
		\todo{Übungsaufgaben Umrechnen? }
		
\end{frame}




\begin{frame}{Programmieren}{Binäre Zahlendarstellung}
	Positive und Negative Darstellungen
	
	% binäre zahlen + und - darstellung

\end{frame}


\begin{frame}[plain]
	Empty Frame
	 
	\begin{block}{This is a Block}
      This is important information
   \end{block}
 
   \begin{alertblock}{This is an Alert block}
   This is an important alert
   \end{alertblock}
 
   \begin{exampleblock}{This is an Example block}
   This is an example 
   \end{exampleblock}
\end{frame}


\begin{frame}{Programmieren}{Binäre Rechnen}
	
	Addition % und Subtraktion

	$0 + 0 = 0$	\\
	$1 + 0 = 1$	\\
	$0 + 1 = 1$	\\
	$1 + 1 = 0 (1 \text{ Übertrag}) \text{ bzw. } 10$	
	
	%Schulrechnen:
	% 19
	%+11
	% 1
	% 30
	 
	\begin{tabular}{cccccccc|c}
		128 & 64 & 32 & 16 & 8 & 4 & 2 & 1 & Dezimal \\ \hline 
		0   & 0  & 0  & 0  & 1 & 0 & 1 & 1 & $8 + 2 + 1 = 11$ \\ 
		0   & 0  & 0  & 1  & 0 & 1 & 0 & 1 & $16 + 4 + 1 = 21$ \\ \hline
		    &    & 1  & 1  & 1 & 1 & 1 &   &  \\
		0   & 0  & 1  & 0  & 0 & 0 & 0 & 0 & $32$ \\
		\hline 
	\end{tabular}
	 
	 
	% binäres rechnen 
	% https://de.wikipedia.org/wiki/Dualsystem
	
	\todo{Bild: Halbaddierer }
	
	Subtraktion:
	Kurz mit negativ darstellung addieren
	asdasd

\end{frame}


\begin{frame}{Programmieren}{Hexadezimal System}
		
	Hexadezimal System

\end{frame}




\begin{frame}{Aufbau von Computern}{Aufbau von Computern}
		
	Es gibt 2 wichtige Komponenten:
	\begin{itemize}
	\item Speicher - Speichert Inhalte
	\item CPU - Erledigt die Arbeit das Arbeitstier
	\end{itemize}
	
	\todo{Bild: Büro Aktenschrank/Schreibtisch/Arbeiter Analogie }
	\todo{Bild: Block Bild }
	
\end{frame}


\begin{frame}{Aufbau von Computern}{Der Speicher}
		
	Der Speicher beinhaltet unterschiedliches:
	\begin{itemize}
	\item Zustand der Rechnungen
	\item Befehle für die CPU
	\end{itemize}
	
	Aufbau und Typen
	\begin{itemize}
	\item Register/Cache/Ram/etc
	\item Zugriff über Addressen
	\end{itemize}
	
	\todo{Bild: Adressen? }

\end{frame}




\begin{frame}{Aufbau von Computern}{Die CPU}
		
	Central Processing Unit
	\begin{itemize}
	\item Liest Befehle ein und bearbeitet sie
	\end{itemize}

\end{frame}


\begin{frame}[fragile]{Aufbau von Computern}{Befehle}
		
	Befehle
	\begin{itemize}
	\item Geben anweisung was zu tun sein soll
	\item simpleste 1 + 1 
	\item Komplexe Dinge werden in einfache zerlegt
	\end{itemize}
	
	Ein Befehl sieht zum Beispiel so aus:
	0100101010101001110101 
	
	Für den Menschen nicht lesbar, deswegen gibt es umschreibungen für menschen
	
	\todo{Bild: Binary to Human? }
	
	\begin{lstlisting}
	add 1, 1
	\end{lstlisting}
	
	Hier ist noch das Problem wohin mit dem Ergebnis von 1 + 1
	was ist wenn man etwas rechnen will wie 1 + 5 / 6 
	Oder wenn der Benutzer etwas eingeben soll 
	
\end{frame}


\begin{frame}{Aufbau von Computern}{Variablen}

	Zustände der Berechnungen 
	Variablen / Register
	
\end{frame}


\begin{frame}{Aufbau von Computern}{Speicher Teil 2}
		
	Speicher Aufteilungen
	\begin{itemize}
	\item Heap
	\item Stack
	\end{itemize}
	
\end{frame}

% jumps
% vergleiche 

% funktionen -> code im pc , PC register

% assembler darstellung mit labeln 

% ansätze eines assemblers programs?
% tokenizer -> parser -> code gen?

\begin{frame}{Aufbau von Computern}{In der Realität}
		
	Ausblick auf technische Details:

	Befehle - Sprache von CPUs
	
	CISC/RISC
	
	ARM/MIPS/x86
	
\end{frame}


\begin{frame}{Aufbau von Computern}{Der ganze PC}
		
	Bekannt
	\begin{itemize}
	\item Speicher
	\item CPU
	\end{itemize}
	
	Was sonst:
	
	\begin{itemize}
	\item HDD
	\item CD/DVD/Blueray/
	\item Grafik-Karte (Bildschirm)
	\item Maus/Tastatur
	\end{itemize}
	 
	 Wie redet alles miteinander
	 Verbindungen: Busse, Mainboard/ South/Northbrige/Sata/PCI Express PCIe/DMA 
	 	-> Memory Mapping
	 	\todo{Bild: Memory Mapping -> Schubladen Tausch }
	 	
	 
	 (Mehrere CPUs -> GPUs ...)
	
	\todo{Bild: Kompletter PC }
\end{frame}


\begin{frame}{Aufbau von Computern}{Wie etwas auf den Bildschirm kommt}
	Einfach:
	
	
	\todo{Bild: Schubladen Tausch / Männlein das mit den Akten ein Fester tapeziert }	
\end{frame}



\begin{frame}{Aufbau von Computern}{Wo Programme laufen}
	OS Betriebssystem
	Container
	Bibliotheken etc
	
	In der Realität läuft der Code in einem speziellem Umfeld.
	
	\todo{Bild: OS }	
\end{frame}


\begin{frame}{Aufbau von Computern}{Zeichencodierungen}
	\todo{Bild: Wo einordnen? }
	
	Text vs Value
	
	Zeichensätze
	'1' -> 67
		
\end{frame}




% praxis: http://courses.missouristate.edu/kenvollmar/mars/
