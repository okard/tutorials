% computer basics

% how a computer works

\begin{frame}{Programmieren}{Wie funktioniert ein Computer}
	Vorab:
	\begin{itemize}
	\item Zahlensysteme
	\end{itemize}	
	
	Wie arbeitet ein Computer:
	\begin{itemize}
	\item Befehle
	\item Speicher
	\item Variablen \& Adressen
	\end{itemize}
\end{frame}


% Zahlen Systeme

\begin{frame}{Programmieren}{Zahlensysteme}

	Wir rechnen im Dezimal System:
	\begin{itemize}
	\item 0,1,2,3,4,5,6,7,8,9,10 
	\item Ab dem Schritt 10  eine Stelle ergänzt(von 9 auf 10) 
	\item Wiederholt sich
	\item Zehnersystem
	\end{itemize}
	
	
	Das selbe funktioniert auch mit anderen Zahlensystemen zum Beispiel 0-3 
	0,1,2,3,10,
	11,12,13,20,
	21,22,23,30
	Vierersystem
	
	Warum erzähl ich das?
	
\end{frame}

%binäre zahle 1
\begin{frame}{Programmieren}{Binäre Zahlendarstellung}

	Der Computer benutzt eine darstellung mit genau 2 Zahlen 0 oder 1 
	Da das ganze elektrisch funktioniert: Strom an oder Strom aus	
	
	0 oder 1
	\emph{an} oder \textit{aus}

	Dualsystem, Zweiersystem	
	
	das ganze fällt unter Digitaltechnik
	Wenn Strom eine andere rolle spielt: Analog
	
\end{frame}

%analog: https://de.wikipedia.org/wiki/Analogrechner

%binäre zahle 2
\begin{frame}{Programmieren}{Binäre Zahlendarstellung}
		1,
		10101
		
		\begin{tabular}{cccccccc|c}
		128 & 64 & 32 & 16 & 8 & 4 & 2 & 1 & Dezimal \\ \hline 
		0   & 0  & 0  & 0  & 0 & 0 & 0 & 1 & 1 \\ 
		0   & 0  & 0  & 1  & 0 & 1 & 0 & 1 & $16 + 4 + 1 = 21$ \\ 
		\hline 
		\end{tabular} 	
\end{frame}

\begin{frame}[plain]
	Empty Frame
\end{frame}


% binäre zahlen + und -

% hex dezimal system

% binäres rechnen 
% https://de.wikipedia.org/wiki/Dualsystem

% befehle

% speicher (register,cache,ram)

% variablen/informationen

%addressen




% Wie funktionieren programmiersprachen?

% wie funktionieren schleifen, verzweigungen, funktionen (parameter)

% Heap & Stack
