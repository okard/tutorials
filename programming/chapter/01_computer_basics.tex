% computer basics

% how a computer works

\begin{frame}{Programmieren}{Wie funktioniert ein Computer}
	Inhalt
\end{frame}


%binäre zahle 1
\begin{frame}{Programmieren}{Binäre Zahlendarstellung}

	Wir rechnen im Dezimal System:
	0,1,2,3,4,5,6,7,8,9,10 
	
	Ab dem Schritt 10 wiederholt sich alles (von 9 auf 10) Zehnersystem
	
	Das selbe funktioniert auch mit anderen Zahlensystemen zum Beispiel 1-3 
	0,1,2,3,10,
	11,12,13,20,
	21,22,23,30

	0 oder 1
	\emph{an} oder \textit{aus}
	
	Digital
	
\end{frame}

%binäre zahle 2
\begin{frame}{Programmieren}{Binäre Zahlendarstellung}
		1,
		10101
		
		\begin{tabular}{cccccccc|c}
		128 & 64 & 32 & 16 & 8 & 4 & 2 & 1 & Dezimal \\ \hline 
		0   & 0  & 0  & 0  & 0 & 0 & 0 & 1 & 1 \\ 
		0   & 0  & 0  & 1  & 0 & 1 & 0 & 1 & $16 + 4 + 1 = 21$ \\ 
		\hline 
		\end{tabular} 	
\end{frame}

%binäre zahlen + und -

%hex dezimal system

%analog: https://de.wikipedia.org/wiki/Analogrechner

%befehle

%variablen/informationen

%speicher

%addressen

