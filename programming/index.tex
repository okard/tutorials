\documentclass[10pt]{beamer}
\usepackage[utf8]{inputenc}
\usepackage[german,english]{babel}
\usepackage{amsmath}
\usepackage{amsfonts}
\usepackage{amssymb}
\usepackage{graphicx}

\usetheme{default}
\usecolortheme{whale}
\usefonttheme{professionalfonts}

\definecolor{turquois}{rgb}{0, 0.3, 0.3}
\definecolor{blue}{rgb}{0, 0.0, 0.8}

\setbeamercolor{title}{bg=turquois}
\setbeamercolor{frametitle}{bg=turquois}

\beamertemplatenavigationsymbolsempty
\setbeamertemplate{footline}[frame number]

\hypersetup{colorlinks,linkcolor=gray,urlcolor=blue}

\title{Programmieren}
\subtitle{Eine Einführung}
%\author{okard}

\begin{document}

\frame{\titlepage}


% introduction

%\begin{frame}[Overlay-Aktionen][Optionen]{Titel}{Untertitel}
%	Inhalt
%\end{frame}

\section{Introduction}

\begin{frame}{Programmieren}{Lizenz}

	\begin{block}{Creative Commons Lizenzvertrag}
     Dieses Werk bzw. Inhalt steht unter einer Creative Commons Namensnennung - Nicht-	kommerziell - Weitergabe unter gleichen Bedingungen 3.0 Unported Lizenz.
	\end{block}
	\begin{center}
	\href{http://creativecommons.org/licenses/by-nc-sa/3.0/deed.de}{$\blacktriangleright$ CC BY-NC-SA}
	\end{center}
\end{frame}


\begin{frame}{Programmieren}{Eine Einführung}
	Was lernt man hier:
	\begin{itemize}
	\item Wie ein Computer arbeitet
	\item Wie Daten in einem Computer aufgebaut sind
	\item Wie Programmiersprachen funktionieren
	\end{itemize}
	\vspace*{0.3cm}
	
	Worauf wird hier nicht näher eingegangen
	\begin{itemize}
	\item Eine echte Programmiersprache \todo{entfernen?}
	\item Genaue technische Details
	\item Warum? -> Später mehr
	\end{itemize}
\end{frame}


\begin{frame}{Programmieren}{Voraussetzungen}
	Zielgruppe: 12+ / ab 7. Klasse

	Notwendig:
	\begin{itemize}
	\item Interesse (auch ein ganz kleines bisschen für Mathe)
	\end{itemize}
	\vspace*{0.3cm}
	
	Von Vorteil:
	\begin{itemize}
	\item Englisch-Grundkenntnisse
	\item Verständnis für Mathematik
	\end{itemize}
\end{frame}

\begin{frame}{Programmieren}{Voraussetzungen}
	Mathematische Thematiken:
	\begin{itemize}
	\item Grundrechenarten ( Addieren, Subtrahieren, Multiplikation)
	\item Zahlensysteme (Dezimal, Binär/Dual-System)
	\item Potenzen ( $2^3$ )
	\end{itemize}
	
	
	Englisch:
	\begin{itemize}
	\item Nur einzelne Wörter
	\item http://dict.leo.org/
	\item http://www.dict.cc/
	\end{itemize}	
\end{frame}


\begin{frame}{Programmieren}{Motivation}
	Muss man das so machen?
		-> Nein
	
	Warum dann aber so detailliert?
		Es geht um das verstehen -> keine Magic Box
		Entmystifizierung des PCs
	
	Problematik:
		Kann am Anfang langweilig sein
		Kann am Anfang kompliziert klingen
			(Ist teilw. kompliziert)
	
	Persönliche Meinung:
		Bringt unheimlich viel
		Muss es nicht alles auswendig wissen
		Grobes Verständnis was im Hintergrund passiert
		Gut gerüstet für alle Programmiersprachen
	
\end{frame}


\begin{frame}{Programmieren}{Background}
	Was ist mein Ziel/Ansprüche
		-> Verständnis Arbeitsweise des PCs
	
	Kritik/Feedback/Anregungen
	
\end{frame}
% computer basics

% how a computer works

\begin{frame}{Programmieren}{Wie funktioniert ein Computer}
	Inhalt
\end{frame}


%binäre zahle 1
\begin{frame}{Programmieren}{Binäre Zahlendarstellung}

	Wir rechnen im Dezimal System:
	0,1,2,3,4,5,6,7,8,9,10 
	
	Ab dem Schritt 10 wiederholt sich alles (von 9 auf 10) Zehnersystem
	
	Das selbe funktioniert auch mit anderen Zahlensystemen zum Beispiel 1-3 
	0,1,2,3,10,
	11,12,13,20,
	21,22,23,30

	0 oder 1
	\emph{an} oder \textit{aus}
	
	Digital
	
\end{frame}

%binäre zahle 2
\begin{frame}{Programmieren}{Binäre Zahlendarstellung}
		1,
		10101
		
		\begin{tabular}{cccccccc|c}
		128 & 64 & 32 & 16 & 8 & 4 & 2 & 1 & Dezimal \\ \hline 
		0   & 0  & 0  & 0  & 0 & 0 & 0 & 1 & 1 \\ 
		0   & 0  & 0  & 1  & 0 & 1 & 0 & 1 & $16 + 4 + 1 = 21$ \\ 
		\hline 
		\end{tabular} 	
\end{frame}

%binäre zahlen + und -

%hex dezimal system

%analog: https://de.wikipedia.org/wiki/Analogrechner

%befehle

%variablen/informationen

%speicher

%addressen



% functions
% datentypen

% wie etwas auf den bildschirm kommt

% menschliche vs computer sprache

% assembler
% hochsprachen
%  interpreter, compiler, 
% programmiersprachen
% paradigmen 

% sprachen? folge tutorials


\end{document}