\documentclass{beamer}

\usepackage[ngerman]{babel}
\usepackage[utf8]{inputenc}

\usetheme{default}
\usecolortheme{whale}
\usefonttheme{professionalfonts}

\definecolor{turquois}{rgb}{0, 0.3, 0.3}
\definecolor{blue}{rgb}{0, 0.0, 0.8}

\setbeamercolor{title}{bg=turquois}
\setbeamercolor{frametitle}{bg=turquois}

\beamertemplatenavigationsymbolsempty
\setbeamertemplate{footline}[frame number]

\hypersetup{colorlinks,linkcolor=gray,urlcolor=blue}


% Include svgs
\newcommand{\executeiffilenewer}[3]{%
\ifnum\pdfstrcmp{\pdffilemoddate{#1}}%
{\pdffilemoddate{#2}}>0%
{\immediate\write18{#3}}\fi%
}
% includesvg[scale]{file} command (linux-version)
\newcommand{\includesvg}[2][1]{%
  \executeiffilenewer{#2.svg}{#2.pdf}{%
  /usr/bin/inkscape -z -D --file="#2.svg" --export-pdf="#2.pdf"}%
  \scalebox{#1}{\includegraphics{#2.pdf}}%
}

%\scalebox{#1}{\input{#2.pdf_tex}

\title{Das Internet}
\author{okard}

\begin{document}


%----------------------
%- Title page
%----------------------
\frame{ \titlepage }

%lizenz
% präsentation / video

\begin{frame}
\begin{figure}
\centering
\includegraphics[keepaspectratio=true,width=0.8\paperwidth]{img/CC-BY-NC-SA_Title.pdf}
\end{figure}

\vspace{3em}
\begin{small}
 \href{http://creativecommons.org/licenses/by-nc-sa/3.0/deed.de}{Creative Commons Namensnennung - Nicht-kommerziell - Weitergabe unter gleichen Bedingungen 3.0 Unported Lizenz.}
 

\end{small}





\end{frame}


% introduction

%\begin{frame}[Overlay-Aktionen][Optionen]{Titel}{Untertitel}
%	Inhalt
%\end{frame}

\begin{frame}{Programmieren}{Eine Einführung}
	Inhalt
\end{frame}

\begin{frame}{Programmieren}{Voraussetzungen}
	Englisch
	Mathe
\end{frame}



\section{Netzwerke}

%siehe auch folie mit themen um was es geht?

%was sind netzwerke?
%was ist ein computer netzwerk

%---------------------------------
\begin{frame}{Computer-Netzwerke}{Übersicht}

	%bild: 
	
	\begin{itemize}
	\item Laptop
	\item PC
	\item Router
	\item (DSL-)Modem
	\end{itemize}
	
	%handy
	%kabel wlan

\end{frame}

%---------------------------------
\begin{frame}{Computer-Netzwerke} {Das Internet}
	Das Internet
\end{frame}


%---------------------------------

%Was ist eine Verbindung

\begin{frame}{Netzwerke}{Verbindungen}
	IP-Addresse
\end{frame}

\begin{frame}{Netzwerke - Verbindungen}
	Ports
\end{frame}


\begin{frame}{Netzwerke - Ports}
	Ports
\end{frame}

\begin{frame}{Netzwerke - Dienste}
	Ports-Dienste
\end{frame}


% Heimnetzwerke

% Öffentliche und Private IP

% NAT

% Port Forwarding

% Protokolle
	% TCP/UDP ?
	% HTTP
	% Email?








\begin{frame}{Computer-Alltag}

\begin{itemize}
\item E-Mail
\item Chatten
\item Surfen
\end{itemize}

\end{frame}

% viren
% trojaner
% würmer
% drive by downloads
% sicherheitslücken
% hacker






\end{document}